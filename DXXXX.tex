\documentclass{wg21}

\usepackage[utf8]{inputenc}
\usepackage{xcolor}
\usepackage{soul}
\usepackage{ulem}
\usepackage{fullpage}
\usepackage{parskip}
\usepackage{csquotes}
\usepackage{listings}
\usepackage{minted}
\usepackage{enumitem}

\lstdefinestyle{base}{
  language=c++,
  breaklines=false,
  basicstyle=\ttfamily\color{black},
  moredelim=**[is][\color{green!50!black}]{@}{@},
  escapeinside={(*@}{@*)}
}

\newcommand{\cc}[1]{\mintinline{c++}{#1}}
\newminted[cpp]{c++}{}


\title{Exploring the design space of metaprogramming and reflection}
\docnumber{DXXXXR0}
\audience{SG 7}
\author{Daveed Vandevoorde}{daveed@edg.com}
\authortwo{Louis Dionne}{ldionne.2@gmail.com}

\begin{document}
\maketitle


%%%%%%%%%%%%%%%%%%%%%%%%%%%%%%%%%%%%%%%%%%%%%%%%%%%%%%%%%%%%%%%%%%%%%%%%%%%%%%
\section{Introduction}
\textit{Metaprogramming} is the craft of creating or modifying programs as the
product of other programs. \textit{Reflective metaprogramming} is metaprogramming
that incorporates data from the program being modified. For example, one can
imagine a metaprogram that automatically generates a hash function from a
user-defined class type: Such a metaprogram would produce the hash function
from meta-information reflected from the user-defined type.

The goal of this paper is to explore the design space for metaprogramming and
reflection in C++. To this end, it is useful to identify the different dimensions
of this design space. We believe those dimensions are:

\begin{enumerate}
  \item Reflection: A way to represent significant aspects of the program as
        first class entities in the program.
  \item Code synthesis: A way to generate new code and/or modify user-written
        code programmatically.
  \item Control flow: A computationally complete mechanism to guide the
        reflection and code synthesis activities.
\end{enumerate}

There are different ways to approach each of these dimensions, and hence there
are multiple different \textit{cuts} we may want to take in our design space.
The rest of this paper explores that, with notes describing strengths and
weaknesses.


%%%%%%%%%%%%%%%%%%%%%%%%%%%%%%%%%%%%%%%%%%%%%%%%%%%%%%%%%%%%%%%%%%%%%%%%%%%%%%
\section{The dimensions}
We start by exploring all the approaches we can currently think of to implement
each dimension described above. Here's a summarized view of the whole thing:

\begin{enumerate}
  \item Reflection
    \begin{itemize}
      \item Type syntax (\cite{P0194})
      \item Heterogeneous value syntax (each \cc{reflexpr()} has a different type) (\cite{P0590})
      \item Homogeneous value syntax (\cc{reflexpr()} always returns \cc{meta::info}) (\cite{P0598})
    \end{itemize}

  \item Code synthesis
    \begin{itemize}
      \item Raw string injection
      \item Token sequence injection
      \item Programmatic API
      \item Metaclasses
    \end{itemize}

  \item Control flow
    \begin{itemize}
      \item Classic template metaprogramming (type syntax)
      \item Heterogeneous value metaprogramming (value syntax, but templates under the hood)
      \item Constexpr programming
    \end{itemize}
\end{enumerate}


%%%%%%%%%%%%%%%%%%%%%%%%%%%%%%%%%%%%%%%%%%%%%%%%%%%%%%%%%%%%%%%%%%%%%%%%%%%%%%
\subsection{Reflection}
This section gives an overview of the different proposed approaches for
supporting reflection. We do not go into details here since the corresponding
proposals already explain this.

\subsubsection{Type syntax (\cite{P0194})}
The result of the reflection operator is meta information in the form of
\textbf{a type that is unique for each entity that we reflect upon}. We
are provided with \cc{type_traits}-like operations to manipulate this meta
information:

\begin{cpp}
struct Foo { int x; float y; };
using MetaInfo = reflexpr(Foo);
using Members = get_data_members_t<MetaInfo>;
\end{cpp}

%%%%%%%%%%%%%%%%%%%%%%%%%%%%%%%%%%%%%%%%%%%%%%%%%%%%%%%%%%%%%%%%%%%%%%%%%%%%%%
\subsubsection{Heterogeneous value syntax (\cite{P0590})}
The result of the reflection operator is meta information in the form of
\textbf{an object whose type is unique for each entity that we reflect upon}.
We are provided with function templates that manipulate this meta information:

\begin{cpp}
struct Foo { int x; float y; };
auto meta_info = reflexpr(Foo);
auto members = get_data_members(meta_info);
\end{cpp}

An important thing to notice here is that the reflection operator really returns
objects of different types depending on what it is reflecting upon:

\begin{cpp}
struct Foo { int x; float y; };
struct Bar { char w; long z; };
auto meta_info_Foo = reflexpr(Foo);
auto meta_info_Bar = reflexpr(Bar);
// The two meta_infos are objects of different types
\end{cpp}

%%%%%%%%%%%%%%%%%%%%%%%%%%%%%%%%%%%%%%%%%%%%%%%%%%%%%%%%%%%%%%%%%%%%%%%%%%%%%%
\subsubsection{Homogeneous value syntax (\cite{P0598})}
The result of the reflection operator is meta information in the form of
\textbf{a constant expression whose type is always \cc{meta::info}}, regardless
of the entity we reflect upon. We are provided with normal constexpr functions
that manipulate this meta information:

\begin{cpp}
struct Foo { int x; float y; };
constexpr meta::info meta_info = reflexpr(Foo);
constexpr std::array<meta::info, 2> members = get_data_members(meta_info);
\end{cpp}

Since the reflection operator returns objects of the same type regardless of
the entity we reflect upon, traditional constexpr programming is applicable.
To make this easier to work with, it would be interesting to expand the
capabilities of constexpr to allow for variable length sequences
(a \cc{constexpr_vector} instead of \cc{array}).


%%%%%%%%%%%%%%%%%%%%%%%%%%%%%%%%%%%%%%%%%%%%%%%%%%%%%%%%%%%%%%%%%%%%%%%%%%%%%%
\subsection{Code synthesis}
This section gives an overview of the different proposed approaches for
supporting code synthesis. To showcase the three first approaches, which
perform direct code synthesis, we solve the toy problem of programmatically
generating a function that applies some function to each member of the
following struct:

\begin{cpp}
struct S {
  char x;
  int y;
  long z;
};
\end{cpp}

The reason why we use this toy problem as opposed to solving the same challenge
in the general case is that solving it in the general case requires control flow,
which we're trying to decouple from code synthesis.

\subsubsection{Raw string injection}
The idea is to allow arbitrary strings to be fed to the implementation:

\begin{cpp}
template<typename Function>
void challenge(S const& s, Function f) {
  constexpr {
    meta::queue(meta::code("f(s.", NAME_OF_X, ");"));
    meta::queue(meta::code("f(s.", NAME_OF_Y, ");"));
    meta::queue(meta::code("f(s.", NAME_OF_Z, ");"));
  }
}
\end{cpp}

Here, \cc{NAME_OF_X} \& friends denote the name of the member of the struct
obtained through a reflection API. Given this, the intent is that \cc{meta::queue}
queues up the string for consumption by the compiler when the constexpr evaluation
completes (i.e., after the closing brace of \cc{constexpr { ... }}). Hence, an
instantiation of \cc{challenge<F>} would expand to

\begin{cpp}
template<>
void challenge<F>(S const& s, F f) {
  { f(s.x); }
  { f(s.y); }
  { f(s.z); }
}
\end{cpp}

Note that we presume the compiler would create a local scope into which each
injected string is translated. This avoids "leakage" from one injection into
another (and from injections into the surrounding context, which would raise
tricky semantic questions).

Also, this example postulates a \cc{constexpr { ... }} construct not seen before,
which forces the compile-time evaluation of some statements. This conceivably be
avoided by moving the body to a constexpr function (or lambda) and calling that
function from a context that guarantees compile-time evaluation (e.g., a constexpr
variable initializer), but expressing this directly seems highly desirable.


%%%%%%%%%%%%%%%%%%%%%%%%%%%%%%%%%%%%%%%%%%%%%%%%%%%%%%%%%%%%%%%%%%%%%%%%%%%%%%
\subsubsection{Token sequence injection}
The idea is to allow "token sequences" to be fed to the implementation, with
a mechanism to parameterize those token sequences:

\begin{cpp}
template<typename Function>
void challenge(S const& s, Function f) {
  constexpr {
    -> { f(s.(.NAME_OF_X.)); }
    -> { f(s.(.NAME_OF_Y.)); }
    -> { f(s.(.NAME_OF_Z.)); }
  }
}
\end{cpp}

Here we assume that \cc{-> { ... }} injects the \cc{{ ... }} part of that
construct after the closing brace of \cc{constexpr { ... }}. Note that we
introduced a notation \cc{(. expression .)} to transform a reflected value
back into an identifier. Other such notations would be required for a few
other entities (e.g. \cc{typename(expression)} to produce a type name). The
expansion would be equivalent to the version using raw string injection.


%%%%%%%%%%%%%%%%%%%%%%%%%%%%%%%%%%%%%%%%%%%%%%%%%%%%%%%%%%%%%%%%%%%%%%%%%%%%%%
\subsubsection{Programmatic API}
The idea is to allow programmatic construction of an intermediate representation
of the generated code, not unlike what is done to the DOM of HTML web pages:

\begin{cpp}
template<typename Function>
void challenge(S const& s, Function f) {
  constexpr {
    meta::queue(
      meta::expr_statement(meta::call_by_name("f",
        meta::select_member(reflexpr(s), NAME_OF_X))));
    meta::queue(
      meta::expr_statement(meta::call_by_name("f",
        meta::select_member(reflexpr(s), NAME_OF_Y))));
    meta::queue(
      meta::expr_statement(meta::call_by_name("f",
        meta::select_member(reflexpr(s), NAME_OF_Z))));
  }
}
\end{cpp}


%%%%%%%%%%%%%%%%%%%%%%%%%%%%%%%%%%%%%%%%%%%%%%%%%%%%%%%%%%%%%%%%%%%%%%%%%%%%%%
\subsubsection{Metaclasses}
Herb Sutter and Andrew Sutton have been exploring a powerful notion of
\textit{metaclasses} (private communication). The idea here is to associate
some metacode to class types using a shorthand notation and have that code run
when the class definition is about to be completed. For example (using a
different notation from what Herb and Andrew are exploring):

\begin{cpp}
constexpr {
  meta::register_metaclass(
    "value",  // Metaclass name
    [](meta::info class_info) {  // constexpr lambda
      ...  // Check and/or modify the class represented by class_info.
           // E.g., diagnose that all members are value types too.
    });
}

class(value) X {
  int x, y;
};  // The constexpr function (or lambda) registered
\end{cpp}

The ability to "attach" metacode in such convenient ways should perhaps be
considered to some degree when evaluating future directions for reflective
metaprogramming.


%%%%%%%%%%%%%%%%%%%%%%%%%%%%%%%%%%%%%%%%%%%%%%%%%%%%%%%%%%%%%%%%%%%%%%%%%%%%%%
\subsubsection{Observations}
Looking at these four approaches, we make the following observations:

\begin{itemize}
  \item Only a metaprogramming API provides a natural path for a "code modification"
        system. For example, one could imagine an API to make a member function
        virtual after it has been declared.
  \item The API approach may require knowing details about the language, like
        expressions and other technical terms that would have to be reflected
        through the API. This may end up being verbose.
  \item The string injection approach could be more powerful than the token sequence
        approach, because strings are first-class values and could therefore be
        composed in more sophisticated ways. On the flip side, that opens the
        possibility of code that is hard to read and debug, because the construction
        of the string eventually injected into the program could be completely opaque.
  \item Although two implementations are known to already have string-injection
        facilities for other reasons, that may not be true of other implementations.
        The ability to replay a sequence of tokens, however, is present in all
        implementations since it is needed to correctly deal with member function
        bodies appearing in class definitions (they have to be stored and then
        replayed after the complete class definition has been seen).
\end{itemize}


%%%%%%%%%%%%%%%%%%%%%%%%%%%%%%%%%%%%%%%%%%%%%%%%%%%%%%%%%%%%%%%%%%%%%%%%%%%%%%
\subsection{Control flow}

\subsubsection{Classic template metaprogramming}
This computation mechanism based on template specialization is well known, and
we have extensive experience using it in libraries like \cite{Boost.MPL}. It
uses an arcane syntax, it is inflexible, and it does not scale properly.

\subsubsection{Heterogeneous value metaprogramming}
This is a newer way of metaprogramming made possible by type deduction facilities
in C++14. This technique uses templates under the hood, just like the classic
template metaprogramming approach, but those are hidden from the users by
throwing a nice value-syntax API on top. This is used extensively in the
\cite{Boost.Hana} library. It works well, but it has some tricky corners
and does not scale in terms of compilation times, just like classic template
metaprogramming.

If we go down this road, Andrew Sutton's for-loop on tuples proposed in
\cite{P0589} and more extended "static" control flow could make things
more palatable.

\subsubsection{Constexpr programming}
Given a way to interoperate between types and \textbf{homogeneous} constant
expressions, we could use the full facilities available during constexpr
programming for metaprogramming. If this can be figured out, it would provide
the most natural way of metaprogramming, because constexpr programming is
simply a subset of normal runtime programming.


%%%%%%%%%%%%%%%%%%%%%%%%%%%%%%%%%%%%%%%%%%%%%%%%%%%%%%%%%%%%%%%%%%%%%%%%%%%%%%
\section{Possible cuts in the design space}
This section describes possible combinations of approaches from each dimension
discussed above to assemble a full solution. We illustrate each solution by
using an example inspired by Andrew Sutton that composes Howard Hinnant's
proposed \cc{hash_append} interface for members of a user-defined type.

\subsection{Classic template metaprogramming with \cite{P0194}}
\begin{itemize}
  \item Reflection: Type syntax
  \item Code synthesis: None needed for runtime, TODO for synthesizing types
  \item Control flow: Classic template metaprogramming
\end{itemize}

Here is an example of what can be achieved with this approach:

\begin{cpp}
template<HashAlgorithm H, SimpleStruct S>
void hash_append(H& h, S const& s) {
  meta::for_each<meta::get_data_members_m<reflexpr(S)>>(
    [&](auto meta_mem) {
      using MetaMem = typename decltype(meta_mem)::type;
      hash_append(h, s.unreflexpr(MetaMem));
    }
  );
}
\end{cpp}

The advantage of this approach is that various parts of the solution are
already available in the language; only a reflection system has to be added.
The major disadvantages are:

\begin{itemize}
  \item Template instantiation is a resource-intensive compiler activity. We
        have extensive experience showing it does not scale sufficiently for
        some heavier use cases (\cite{Boost.MPL}).
  \item Template metaprogramming uses arcane notations and conventions when
        compared to plain C++ programming.
\end{itemize}


%%%%%%%%%%%%%%%%%%%%%%%%%%%%%%%%%%%%%%%%%%%%%%%%%%%%%%%%%%%%%%%%%%%%%%%%%%%%%%
\subsection{Heterogeneous value style with \cite{P0590}}
\begin{itemize}
  \item Reflection: Heterogeneous value syntax
  \item Code synthesis: None needed for runtime, TODO for synthesizing types
  \item Control flow: Heterogeneous value metaprogramming
\end{itemize}

This approach is very similar to the one above, but we use objects instead of
types, which allows us to stay closer to the usual syntax used in C++. Here's
what can be achieved with the help of the Hana library (assuming heterogeneous
value-based reflection):

\begin{cpp}
template<HashAlgorithm H, SimpleStruct S>
void hash_append(H& h, S const& s) {
  hana::for_each(reflexpr(S).member_variables(), [&](MemberVariable v) {
    hash_append(h, s.*(v.pointer()));
  });
}
\end{cpp}

\cc{hana::for_each} is an algorithm working on tuples that basically calls the
provided lambda with each element of the tuple, in order. Note that
\cc{MemberVariable} is not a traditional type, but a concept name (allowing
each expansion of the body to occur for different types). However, we could
mitigate the notational burden by adopting a construct to expand code over the
element of a tuple (and, potentially, tuple-like entities such as packs), as
proposed in \cite{P0590}. Note that this construct is useful independently of
reflection. Here's how the example looks with this construct:

\begin{cpp}
template<HashAlgorithm H, SimpleStruct S>
void hash_append(H& h, S const& s) {
  for... (MemberVariable v : reflexpr(S).member_variables()) {
    hash_append(h, s.*(v.pointer()));
  }
}
\end{cpp}

This is more readable, but it still involves a template-like expansion. The
\cc{for...} construct isn't a loop, but produces a repeated "instantiation"
(or sorts) of its body for various values of \cc{v}, just like \cc{hana::for_each}
does. This approach suffers from the following disadvantages:

\begin{itemize}
  \item Template instantiation is a resource-intensive compiler activity. We
        have experience showing it does not scale sufficiently for some heavier
        use cases (\cite{Boost.Hana}).
  \item Unless we introduce the ability to change the type of an object, the
        fact that we're manipulating heterogeneous objects means that we can't
        perform mutations. While this has some nice properties, it strays away
        from what we're used to in C++.
\end{itemize}


%%%%%%%%%%%%%%%%%%%%%%%%%%%%%%%%%%%%%%%%%%%%%%%%%%%%%%%%%%%%%%%%%%%%%%%%%%%%%%
\subsection{Homogeneous value syntax with \cite{P0598}}
\begin{itemize}
  \item Reflection: Homogeneous value syntax
  \item Code synthesis: Can use anything
  \item Control flow: Constexpr programming
\end{itemize}

Another possibility is to allow metaprograms to be written in "plain C++".
Running "plain C++" at compile time is already possible today through
\cc{constexpr} evaluation. Plain C++ operates primarily on values and
therefore it makes sense to reflect program information into values rather
than types. In theory, we could imagine a handful of reflection types to
reflect different entities (e.g., a type to reflect functions, another to
reflect types, yet another for namespaces, etc.), but it is easier to deal
with a single reflection type -- say \cc{meta::info} -- so that a collection
of reflection values can easily be used to represent heterogeneous program
information (e.g., a template argument list, which may contains types,
constants, and templates). That also means we'd want a dynamic container
type usable during constexpr evaluation (e.g., \cc{constexpr_vector<T>} as
described in \cite{P0597}), which is a facility useful on its own (i.e., not
just in the context of reflective metaprogramming).

A skeleton of the \cc{hash_append} example might thus look like the following:

\begin{cpp}
template<HashAlgorithm H, SimpleStruct S>
void hash_append(H& h, S const& s) {
  constexpr {
    for (meta::info member: meta::data_members(reflexpr(s))) {
      <generate a hash_append call for "member">
    }
  }
}
\end{cpp}

In this example, \cc{reflexpr(s)} produces a value of a unique reflection type
\cc{meta::info} and \cc{meta::data_members(reflexpr(s))} then produces a value
of type \cc{constexpr_vector<meta::info>} which can be iterated over. Note that
those values can be discarded by the compiler when it is done evaluating the
code. For the generation of the code in the body, we can use any of the direct
code synthesis approaches. Raw string injection:

\begin{cpp}
template<HashAlgorithm H, SimpleStruct S>
void hash_append(H& h, S const& s) {
  constexpr {
    for (meta::info member : meta::data_members(reflexpr(s))) {
      meta::queue(meta::code("hash_append(h, s.", meta::name(member), ");"));
    }
  }
}
\end{cpp}

Token sequence injection:

\begin{cpp}
template<HashAlgorithm H, SimpleStruct S>
void hash_append(H &h, S const &s) {
  constexpr {
    for (meta::info member : meta::data_members(reflexpr(s))) {
      -> { hash_append(h, s.(.member.)); }
    }
  }
}
\end{cpp}

Programmatic API:

\begin{cpp}
template<HashAlgorithm H, SimpleStruct S>
void hash_append(H& h, S const& s) {
  constexpr {
    for (meta::info member : meta::data_members(reflexpr(s))) {
      meta::queue(
        meta::expr_statement(
          meta::call_by_name(
            "hash_append",
            reflexpr(h),
            meta::select_member(reflexpr(s), member))));
    }
  }
}
\end{cpp}


\section{Discussions and notes}
\subsection{About in-place program transformations}
The ability to modify attributes of a declaration after the declaration has been
completed is an attractive one, but it runs into many specific implementation
difficulties. Most C++ front ends produce mostly-immutable representations of
declarations (this is true, at least, for GCC, Clang, and EDG, but probably all
others as well). For example, it may be very difficult to have turn a member
function that is virtual by virtue of its declaration become non-virtual
"a posteriori" through metaprogramming (however, the reverse transformation
may well be less problematic).

As noted earlier, an API to perform such transformations is one of the few viable
routes to such an ability. Because of the practical limitations on the sorts of
transformations that can be handled by mainstream implementations it seems prudent
to make such an API be very specific to the transformations that are found to be
useful and feasible. For example, perhaps we could envision a \cc{meta::make_member_virtual(...)}
function, but leave out a hypothetical \cc{meta::make_member_nonvirtual(...)}.

Since actual synthesis of code is more cumbersome with the API approach, we can
also consider the possibility of a compact code injection mechanism augmented
(possibly at a later stage) with a limited program transformation API.

\subsection{About injection points}
When metacode synthesizes ("injects") code, that code must end up in some
specific context and scope. When template instantiation is the synthesis
mechanism, those contexts correspond to the points-of-instantiation for the
templates being expanded:

\begin{enumerate}
  \item Namespace scopes for ordinary function template instances
  \item Class scopes for member function template instances
  \item Somewhat more local scopes for generic lambda instances
\end{enumerate}

For more direct approaches to code synthesis, we will want correspondingly
direct ways of specifying where the injected code ends up. In our examples
so far, the injected code always ended up in a local scope. However, it would
not be hard to extend the kinds of notations introduced above to target other
scopes. For example:

\begin{cpp}
template<HashAlgorithm H, SimpleStruct S>
void generate_default_hash_function() {
  meta::info hash_type ...  // Compute some meta information.
  -> :: {  // Inject in file scope
    namespace MyLib {
      template<> typename(hash_type) hash(...) {
         ...
      }
    }
  }
}
\end{cpp}

or

\begin{cpp}
template<SimpleStruct S>
void generate_helper() {
  // ...
  -> namespace {  // Inject in nearest namespace scope.
    void helper(...) { ... }
  }
}
\end{cpp}

or

\begin{cpp}
template<SimpleStruct S>
void generate_print_raw_member() {
  // ... Compute some meta information.
  -> class {  // Inject in class scope.
    void print_raw() {
      generate_helper<...>();  // Cascading synthesis.
    }
  }
}
\end{cpp}

The last option, generating in class scopes, is interesting but also challenging
when dealing with class templates. Consider the following example:

\begin{cpp}
template<typename T> struct X {
  constexpr {
    generate_mem();  // Generates a member "mem" for X<T>.
  }
  void f() {
    mem();  // Error? ("mem" not found.)
  }
};
\end{cpp}

Here we added some metacode to insert a member \cc{mem} in every \cc{X<T>}.
Such a member is likely to want to depend on the specific type \cc{T} for
which \cc{X} is instantiated. That argues for running the metacode at
instantiation time, but that means that in the generic version there is
no member \cc{mem}, which may limit the usefulness of the added member
(e.g., another member cannot reference it). One possible approach to this
issue is to distinguish code injected when the template definition is first
encountered from code injected when it is instantiated:

\begin{cpp}
template<typename T> struct X {
  constexpr {  // Injected in template definition
    -> class { void mem(); }
  }
  constexpr<> {  // Injected in template instantiation
    generate_mem();  // Generates a member "mem" for X<T>.
  }
  void f() {
    mem();  // Error? ("mem" not found.)
  }
};
\end{cpp}


\section{References}
\renewcommand{\section}[2]{}%
\begin{thebibliography}{9}

  \bibitem[P0597]{P0597}
    Daveed Vandevoorde,
    \emph{\cc{std::constexpr_vector<T>}}\newline
    \url{http://www.open-std.org/jtc1/sc22/wg21/docs/papers/2017/p0597r0.html}

  \bibitem[P0598]{P0598}
    Daveed Vandevoorde,
    \emph{Reflect through values instead of types}\newline
    \url{http://www.open-std.org/jtc1/sc22/wg21/docs/papers/2017/p0598r0.html}

  \bibitem[P0590]{P0590}
    Andrew Sutton \& Herb Sutter,
    \emph{A design for static reflection}\newline
    \url{http://www.open-std.org/jtc1/sc22/wg21/docs/papers/2017/p0590r0.pdf}

  \bibitem[P0385]{P0385}
    Matus Chochlík, Axel Naumann \& David Sankel,
    \emph{Static reflection: Rationale, design and evolution}\newline
    \url{http://www.open-std.org/jtc1/sc22/wg21/docs/papers/2017/p0385r2.pdf}

  \bibitem[P0194]{P0194}
    Matus Chochlík, Axel Naumann \& David Sankel,
    \emph{Static reflection}\newline
    \url{http://www.open-std.org/jtc1/sc22/wg21/docs/papers/2017/p0194r3.html}

  \bibitem[Boost.Hana]{Boost.Hana}
    Louis Dionne,
    \emph{Boost.Hana, A modern metaprogramming library}\newline
    \url{https://github.com/boostorg/hana}

  \bibitem[Boost.MPL]{Boost.MPL}
    Aleksey Gurtovoy \& David Abrahams,
    \emph{Boost.MPL}\newline
    \url{http://www.boost.org/doc/libs/release/libs/mpl/doc/index.html}

  \bibitem[P0589]{P0589}
    Andrew Sutton,
    \emph{Tuple-based for loops}\newline
    \url{http://open-std.org/JTC1/SC22/WG21/docs/papers/2017/p0589r0.pdf}

\end{thebibliography}

\end{document}
